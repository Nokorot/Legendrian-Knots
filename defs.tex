

\newcommand{\figpdf}[3]{
\begin{figure}
\centering
\includegraphics[width=#1\textwidth]{figs/#2.pdf}
\caption{#3}
\label{fig:#2}
\end{figure}}

% Specific definitions
\newcommand{\Ainf}{$A_\infty$}

\def \I {\mathbbm{1}}
\def \M {\mathcal{M}}

\newcommand{\wddef}[1]{\underline{#1}}
\newcommand{\<}[1]{\langle #1 \rangle}

\newcommand{\cupS}{\:\rotatebox[origin=c]{180}{$\Pi$}\:}

\newcommand{\ato}{\dashrightarrow}

\newcommand{\Cc}{\mathcal{C}}


\newcommand{\pref}[1]{(\ref{#1})}

\newcommand{\smaletilde}[1]{\tilde{#1}}
\renewcommand{\tilde}[1]{\widetilde{#1}}

\def \d{ \mathrm{d} }

% TODO: 
%\renewcommand{\circ}{\circ\:}

% Standard setts.
\def \N {\mathbb{N}}
\def \Z {\mathbb{Z}}
\def \Q {\mathbb{Q}}
\def \R {\mathbb{R}}
\def \C {\mathbb{C}}

% Vector calculus.
\newcommand{\dif}[3][]{
	\ensuremath{\frac{d^{#1} {#2}}{d {#3}^{#1}}}}
\newcommand{\pdif}[3][]{
	\ensuremath{\frac{\partial^{#1} {#2}}{\partial {#3}^{#1}}}}

\renewcommand{\dd}[1]{\dif{}{#1}}
\newcommand{\pdd}[1]{\pdif{}{#1}}

% Vectors and matrices.
\newcommand{\mat}[1]{\begin{matrix} #1 \end{matrix}}
\newcommand{\pmat}[1]{\begin{pmatrix} #1 \end{pmatrix}}
\newcommand{\bmat}[1]{\begin{bmatrix} #1 \end{bmatrix}}

% Add space around the argument.
\newcommand{\qq}[1]{\quad#1\quad}
\newcommand{\q}[1]{\:\:#1\:\:}

\newcommand{\lskip}{\vspace{\baselineskip}}

% Implications
\newcommand{\la}{\ensuremath{\Longleftarrow}}
\newcommand{\ra}{\ensuremath{\Longrightarrow}}
\newcommand{\lra}{\ensuremath{\Longleftrightarrow}}

\newcommand{\pwf}[1]{\begin{cases} #1 \end{cases}}
\newcommand{\tif}{\text{if}\quad}
\newcommand{\tand}{\text{and}}


% Shorthand
\newcommand{\vphi}{\varphi}
\newcommand{\veps}{\varepsilon}

% Maths Operators
\let\ker\undefined
%\let\span\undefined

\DeclareMathOperator{\id}{id}
\DeclareMathOperator{\pr}{pr}
\DeclareMathOperator{\im}{im}
\DeclareMathOperator{\ker}{ker}
\DeclareMathOperator{\spann}{span}


\DeclareMathOperator{\Diff}{Diff}

% might want to have this is envs.sty or something
% Maths Environments 

% \theoremstyle{plain}
% \newtheorem{thm}{Theorem}[chapter] % reset theorem numbering for each chapter
% 
% \theoremstyle{definition}
% \newtheorem{defn}[thm]{Definition} % definition numbers are dependent on theorem numbers
% \newtheorem{exmp}[thm]{Example} % same for example numbers


\theoremstyle{plain}
\newtheorem{them}{Theorem}[section]
\newtheorem{prop}[them]{Proposition}
\newtheorem{corol}[them]{Corollary}
\newtheorem{lemma}[them]{Lemma}

\newtheorem{defn}[them]{Definition}
\newtheorem{exmp}[them]{Example}
\newtheorem{clame}[them]{Clame}

\theoremstyle{definition}
\newtheorem{remark}[them]{Remark}
