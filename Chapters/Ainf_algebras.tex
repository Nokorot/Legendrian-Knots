
\subsection{Definition}

\begin{defn}
Let $k$ be a field and $\Gamma = \Z/c\Z$ be a cyclic group, for some
$c\in \Z$. An \wddef{\Ainf-algebra over $k$} is a $\Gamma$-graded 
vector space
\[ A = \bigoplus_{p\in\Gamma} A^p  \]
endowed with a family of graded $k$-linear maps
\[ m_n: A^{\otimes n} \to A, n \ge 0, \]
of degree $2-n$ (mod $c$) satisfying the \Ainf relations:

\begin{equation}
\label{eq:Ainf_rels}
\sum_{r+s+t = n} (-1)^{r+st} m_u(\I^{\otimes r} \otimes m_s \otimes \I^{\otimes t}) = 0.
\end{equation}
Where the sum runs over all $r,s,t \ge 0$, such that $r+s+t=n$ and $u = r+1+t$. 
\end{defn}

\subsubsection{Remark:}
Often this definition is known as a \wddef{curved} \Ainf-algebra, where $m_0$ is the \wddef{curvature} of $A$. If $m_0 = 0$ we say $A$ is \wddef{uncurved}.
If $A$ is uncurved, the first \Ainf relation
(ie. $n=1$), demands that $m_1 \circ m_1 = 0$ and we may define the \wddef{cohomology}
$H^*(A,m) = \ker(m_1)/\im(m_1)$, a graded $k$-vector space.
The second \Ainf relation is
\[ m_2(m_2 \otimes \I + \I \otimes m_2), \]
so $m_2$ defines a product on the \wddef{cohomology}, which by the third relation if associative.

In this notes we will, by an \Ainf-algebra, refer to the more general
case of a curved \Ainf-algebra.

\subsubsection{Notation}
Let
\[ TA := \bigoplus_{n\in\Z} A^{\otimes n} \]
denote the tensor-algebra generated by $A$, with the canonical product 
\[ w: TA\otimes TA \to TA; \: (v,w) \mapsto v\otimes w. \]
Then the family $m_n: A^{\otimes n} \to A$ may be writen as a single map $m: TA \to
A$. Throughout this paper we will write an \Ainf-algebra as a pair $(A,m)$, where 
$m: TA \to A$.

If $a \in A$, them we will write $|a| \in \Gamma$ to denote the degree of $a$.

% Note that if we have $a_1,...,a_n \in A$, the relation above reads 
% \begin{equation}
% \label{eq:rel_on_a}
% \xi_n(a_1, ... a_n) =
% \sum_{r+s+t = n} m_u(a_1,..., a_r, m_s( a_{r+1}, ..., a_{r+s} ), a_{r+s+1},
% ..., a_n) = 0. 
% \end{equation}

\begin{defn}
An \Ainf-algebra $(A,m)$ is \wddef{finite} if there exists $N$ such that
\[ m_n = 0, \q{\text{for all}} n\ge N.  \]
\end{defn}

\subsection{Morphisms}

\begin{defn}
Let $(A,m^A)$ and $(B,m^b)$ be \Ainf-algebras. An \wddef{\Ainf-homomorphism} 
$f:A \ato B$ is a family of graded $k$-linear maps
\[ f_n : A^{\otimes n} \to B, n\ge 0 \]
of degree $1-n$ such that

\begin{equation}
\label{eq:ainf_morph_rel}
\sum (-1)^{r+st} f_u(\I^{\otimes r} \otimes m^A_s \otimes \I^{\otimes t}) 
= \sum (-1)^s m^B_r (f_{i_1} \otimes f_{i_2} \otimes ... \otimes
f_{i_r}),
\end{equation}
where the first sum runs over all decompositions $n = r+s+t$, like in
(\ref{eq:Ainf_rels}), and the second sum runs over all $1 \le r \le n$ and all
decompositions $n = i_1 + ... + i_r$. The sign on the right hand side is given
by
\[s = \sum_{j=1}^{r-1} (r-j)(i_j-1).  \]
\end{defn}

$f_0$ is called the \wddef{curvature} of $f$ and if $f_0 = 0$ we say $f$ is
\wddef{uncurved}.
If $(A,m^A)$, $(B,m^B)$ and $f$ are all uncurved, then the first \Ainf-relation is 
\[ f_1 \circ m^A_1 = m^B_1 \circ f_1. \]
So $f_1$ induces a linear map between the cohomologies
\[ f^* : H^*(A, m^A) \to H^*(B,m^B). \]
Note that the further conditions demands that the induced linear map, also
conserves the other algebraic structure on the cohomology. In particular the
second relation ($n=2$), implies that 
\[ f^* \circ m_2^* = m_2(f^* \otimes f^*), \]
where $m_2^*$ is the product induced by $m_2$. So the linear map is in
particular a homomorphism of algebras.

\begin{exmp}
If $(A, m)$ is an \Ainf-algebra, the identity $\id^A : A \ato A$ given by
\[ \id^A_i = \pwf{ \id &\tif i = 1 \\ 0 &\tif i \ne 1, } \]
is an \Ainf-homomorphism.
\end{exmp}

\begin{defn}
If $(A,m^A)$ and $(B,m^B)$ are finite and $f:A \ato B$ is an \Ainf-homomorphism, we say $f$ is \wddef{finite} if there exists $N$, such that $f_n = 0$ for all $n > N$.

If $N = 1$ and $f_0 = 0$, we say $f$ is \wddef{strict}.
\end{defn}

\begin{defn}
\label{def:Ainf_comp}
The \wddef{composition} of two \Ainf-homomorphisms $f: B \ato C$ and $g: A \ato B$ is given
by
\begin{equation}
\label{eq:ainf_comp_rel}
(f \circ g)_n := \sum (-1)^s f_r \circ (g_{i_1} \otimes ... \otimes
g_{i_r}),
\end{equation}
where the sum and sign are the same as in the defining identity.
\end{defn}

\begin{lemma}
The composition $f \circ g: A \ato C$, defines a new \Ainf-homomorphism and if both
$f$ and $g$ are finite, so is 
\end{lemma}

\begin{proof}
Through a simple swap of sums it is relatively easy to show that $f\circ g$,
indeed satisfy (\ref{eq:ainf_morph_rel}). 

Suppose $N\in \N$, is such that $f_n = 0$ and $g_n = 0$ for all $n > N$, then if
$n > N^2$, either $r > N$ or at least one $i > N$. So $(f \circ g)_n = 0$.
\end{proof}

\begin{defn}
An \Ainf homomorphism $f: A\ato B$, is an \wddef{isomorphism} if there exists an 
\Ainf-homomorphism $g: B\ato A$, such that \[ g \circ f = \id^A \qq\tand f \circ g = \id^B. \]
\end{defn}

\begin{lemma}
\label{prop:Ainf_iso_induced}
Let $(A,m^A)$ and $(B,m^b)$ be uncurved \Ainf-algebras. If $f: A \ato B$ is
an \Ainf-isomorphism, then the induced linear map $f^*: H^*(A,m^A) \to
H^*(B,m^B)$ is also an isomorphism of graded vector spaces.
\end{lemma}


\todo{move}

\begin{lemma}
\label{lemma:Ainf_struct_from_pre_hom}
Let $(A,m)$ be an \Ainf-algera and $f : A \ato A$ an \Ainf-pre-homomorphism,
such that $f_1$ is an isomorphism. Then there exists a unique \Ainf-structure
$\tilde{m}$ such that $f$ is an \Ainf-homomorphism from $(A,m)$ to
$(A,\tilde{m})$.
\end{lemma}

\begin{proof}

\[ \tilde{m}_k (f_1(a_1),...,f_1(a_k)) = P(f_i, \tilde{m}_{\le k-1} \}, m_j).  \]

\end{proof}




\begin{proof}
By definition, $(f\circ g)_1 = f_1 \circ g_1$. So $(f \circ g)^* = f^* \circ
g^*$. Hence $f^* \circ g^* = \id_{H^*(A,m^A)}$ and $g^* \circ f^* =
\id_{H^*(B,m^B)}$ and thus $f^*$ is invertible. Since $f_1$ has degree $0$,
so does $f^*$.
\end{proof}

\subsection{Tame morphisms}
\begin{defn}
Let $(A,m^A)$ be an \Ainf-algebra, where $A = k\<{a_1,...,a_n}$. An \Ainf-automorphism $f: A \ato A$ is called
\wddef{elementary} if there exist $j \in \{1,...,n\}$ and $u: TA \to k$, such
that for all $m \in \N$
\[ u_m(...,a_j,...) = 0 \q\tand f_m(a_{i_1},...,a_{i_m}) = \id^A + a_j u(a_{i_1},...,a_{i_m}), 
\]
A finite composition of such \Ainf-automorphisms is called \wddef{tame automorphism}.
A \wddef{tame \Ainf-isomorphism} $f: A \ato B$ is the composition of a tame automorphisms and a strict isomorphisms. 
\end{defn}
\subsection{Stabilization}

\begin{defn}
Let $(A,m)$ be an \Ainf algebra. The \wddef{$i$-th stabilization} of $A$, is a
new \Ainf-algebra $\qty(S_i(A), m^{S_i(A)})$, where
\[ S_i(A) = A \oplus k\<{e_1, e_2}, \]
$|e_1| = i$, $|e_2| = i+1$, 
\[ m_1(e_1) = e_2, \quad m_1(e_2) = 0, \quad m_n(..., e_j, ...) = 0 \quad \forall n \ne 1 \]
and $m^{S_i(A)} (a_1,...,a_n) = m_n(a_1,...,a_n)$ for all $n\in \N$ and $a_1,...,a_n \in A$.
\end{defn}

\begin{defn}
Let $(A,m^A)$ and $(B,m^B)$ be \Ainf-algebras, then they are sad to have the same \wddef{stable type} if there exist 
$i_1,...,i_k,j_1,...,j_l \in \Z$ and a tame \Ainf-isomorphism
\[ f: S_{i_1} (...(S_{i_k} (A, m^A))...) \to S_{j_1} (...(S_{j_l} (B, m^B))...). \] 
\end{defn}

Let $\sigma: S_i(A) \ato A$ and $\tau : A \ato S_i(A)$, such that $\sigma_k
= \tau_k = 0$ for $k> 1$, and 
\[ \tau_1(a) = a, \sigma_1(a) = a, \q\tand \sigma_1(e_j) = 0, \]
for all $a \in A$ and $j = 1,2$. ie. $\tau$ is the inclusion of $A$ in
$S_i(A)$ and $\sigma$ is the projection of $S_i(A)$ onto $A$.


\todo{ This is not done !!! }


\begin{lemma}
\label{lemma:tau_id_homotopy}
There exists a graded linear map $h: TS_i(A) \to TS_i(A)$, such that 
\[ \tau + \id_{S_i(A)} = h \circ m^{S_i(A)} + m \circ h\]

$\tau \circ \sigma$ is homotopic to $\id_A'$ as \Ainf-hom. 
\end{lemma}





