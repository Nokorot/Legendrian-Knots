
% \subsection{The Chekanov-Poincar\'{e} polynomials}

Consider a finite \Ainf-algebra $(A,m)$ over $\Z_2$. If $A$ in uncurved, recall that we may define the cohomology 
\[ H^*(A, m_1) = \frac{\ker m_1}{\im m_1}, \] 
a $\Gamma$ graded $k$-vector space. 

\begin{defn}
If $A$ is uncurved and finite-dimensional then so is the cohomology, so we may define the \wddef{Poincar\'{e} Polynomial}
\[ P_A(\lambda) := \sum_{i \in \Gamma} \dim\qty( H^i(A,m_1) )\cdot \lambda^i. \]
\end{defn}

This is clearly a invariant of the cohomology of uncurved \Ainf algebras. However
is is not clear how this is useful as an invariant of Legendrian knots, as the
associated \Ainf algebra in general is curved.



% \subsection{Maurer-Cartan element}

\begin{defn}
Let $a \in A$, then, for $n \in \N$, define 
$ m^c: TA \to A, $by 
\begin{equation}
\label{eq:def_id_of_mc}
m^c_n(a_1,...,a_n) := \sum_{i_0,...,i_n}
m_u\big(\underbrace{c,...,c}_{i_0},a_1,\underbrace{c,...,c}_{i_1},a_2,c,...,c,a_n,\underbrace{c,...,c}_{i_n}\big), 
\end{equation}
where the sum runs over all combinations of $i_0,...,i_n \ge 0$.
\end{defn}

Note that, since $A$ is finite, the sum is well-defined.
Also note that from here on, we will write $c...$, to mean a sequence of $c$'s. If we write $c,...,c$, the dots might also contain more then
just $c$'s.

\begin{lemma}
\label{prop:Ac_is_ainf}
If $|c| = 1$, then $A^c := (A, m^c)$ defines a new finite curved \Ainf-algebra
\end{lemma}

\begin{proof}
First we need $|c| = 1$, in order for the maps to have the correct degree. With
$|c|=1$ this works out, since the degree increases by 1 for each occurrence of $c$
and decreases by 1 for each extra entry in $m_u$.  
It is immediately clear from the definition that it is finite, so it suffuses to
check that the \Ainf-relations hold.

\begin{align*}
&\sum_{r,s,t} m^c_u (a_1,...,a_r, m^c_s(a_{r+1},...,a_{r+s}), a_{r+s+1},...,a_{n}) \\
&= \sum_{r,s,t} \sum_{i_0,...,i_s} m^c_u (a_1,...,a_r, 
m_{u'}(\underbrace{c...}_{i_{0}},a_{r+1},c,...,c,a_{r+s},\underbrace{c...}_{i_{s}})
a_{r+s+1},...,a_{n})\\
&= \sum_{r,s,t} \sum_{i_0,...,i_s} \sum_{j_0,...,j_{u}} m_{u''} (\underbrace{c...}_{i_0},a_1,c,...c,a_r,\underbrace{c...}_{j_r}, \\
& \qquad\qquad 
m_{u'}(\underbrace{c...}_{i_0},a_{r+1},c,...,c,a_{r+s},\underbrace{c...}_{i_{r+s}}),
\underbrace{c,...}_{j_{r+1}},a_{r+s+1},c,...,c,a_{n},\underbrace{c...}_{j_u}) \\
% &= \sum_{N\ge n} \sum_{r,s,t} \sum_{i_*,j_*} \qty\{\text{same as in the previous line} \}
\end{align*}
In the first line we expand the inner $m^c$ according to
(\ref{eq:def_id_of_mc}), and in the second line we expand the outer $m^c$. The
indices $r,s,t$ are as in the \Ainf-relations (\ref{eq:Ainf_rels}) and the
indices $i_0,...,i_s$ and $j_0,...,j_u$ are as in (\ref{eq:def_id_of_mc}).
Finally $u=r+t+1$, $u' = s + \sum i_k$ and $u'' = r+t+\sum j_k$. (Note that, for
now on, this indices will be assumed without stated explicitly. ) 

Considering the sums in the last line, we can sort them according to the total
number of $c$'s (ie. $N = \sum_k i_k + \sum_k j_k)$. 
\[ \sum_{r,s,t} \sum_{i_0,...,i_s} \sum_{j_0,...,j_{u}} \{ ... \}
= \sum_{N \ge 0} \sum_{r,s,t} \sum_{i'_0,...,i'_s} \sum_{j'_0,...,j'_u} \{
... \}. \]
Then for each $N$ the inner sum is the left hand side of the $N$'th \Ainf-relation in $m$ applied to $\xi_N \in A^{\otimes N}$, the sum of all ways of putting $N$ $c$'s between the $a$'s. ie.
\[  \xi_N = \sum_{k_0,...,k_n} (\underbrace{c...}_{k_0}, a_1, c,...,c, a_n,
\underbrace{c...}_{k_n}),
\qquad k_0+...+k_n = N. \] 
Since the $m$ satisfy the \Ainf-relation, we are done.
\end{proof}

\begin{defn}
An element $c \in A$ is called a \wddef{Maurer-Cartan element} (MC element) if $|c| = 1$ and 
\[ \sum_{n \ge 0} m_n(c...) = 0. \]
\end{defn}

\begin{lemma}
$(A,m^c)$ is uncurved if and only if $c$ is a MC element.
\end{lemma}

\begin{proof}
By definition,
\[ m_0^c = \sum_{n \ge 0} m_n(c...) = 0 \]
\end{proof}

\begin{defn}
If $A$ is a finite-dimensional finite curved \Ainf-algebra, define 
\[ I(A) := \{ P_{A^c} \q: c \in A \text{ is a MC element} \}. \]
\end{defn}

We later will see that this set of Poincar\'{e} polynomials is invariant under
stable tame isomorphism and thus Legendrian isotopy. Not that the isomorphism
need not actually be tame, for this to be true. 

\begin{defn}
\label{def:hom_c}
Let $c \in A$ and $f: A \ato B$, then like in (\ref{eq:def_id_of_mc}), define
$f_n : A^{\otimes n} \to B$ by
\begin{equation}
\label{eq:hom_c}
f^c_n(a_1,...,a_n) := \sum_{i_0,...,i_n}
f_u(\underbrace{c...}_{i_0},a_1,c,...,c,a_n,\underbrace{c...}_{i_n}).
\end{equation}
Also, we write $f_*(c) := f_0^c = \sum_{n\ge 0} f_n(c...)$.
\end{defn}

\begin{lemma}
If $f: A \ato B$ and $c\in A$ is a MC element, then so is $f_*(c) \in B$.
\end{lemma}
\begin{proof}
By the defining relation of $f$ (\ref{eq:ainf_morph_rel}) and linearity we
have.
\begin{align*}
m^B_n(f_*(c), ..., f_*(c)) 
&= \sum_{i_1,...,i_n} m^B_n(f_{i_1}(c...), ..., f_{i_n}(c...)) 
= \sum_{r,s,t} f_u(c..., m_s(c...), c...)  \\
&= \sum_{r,t} f_u(c..., \sum_s m_s(c...), c...)  = 0
\end{align*}
\end{proof}

\begin{lemma}
$\{ f^c_n \}_{n \ge 1}$ defines a finite uncurved \Ainf-homomorphism 
\[ f: A^c \to B^{f_*(c)}. \]
\end{lemma} 

\begin{proof}
We need to check that $f$, satisfy equation (\ref{eq:ainf_morph_rel}). 
We have
\begin{align*}
& \text{LHS} (a_1,...,a_n) \\
&= \sum_{r,s,t} f^c_u (a_1, ..., a_r, m^{A,c}_s(a_{r+1}, ..., a_{r+s}),
a_{r+s+1}, ..., a_{n}) \\
%
% &= \sum_{r,s,t} \sum_{i_1,...,i_s} f^c_u (a_1, ..., a_r, m_s(c...,a_{r+1}, c,
% ...,c, a_{r+s},c...),
% a_{r+s+1}, ..., a_{n}) \\
%
&= \sum_{r,s,t} \sum_{i_0,...,i_s}\sum_{j_0,...,j_u} 
f_u \big(\underbrace{c...}_{j_0},a_1,c,...,c, a_r,\underbrace{c...}_{j_r}, \\
& \qquad\qquad\qquad\qquad\qquad 
m^A_s(\underbrace{c...}_{i_0}, a_{r+1}, c,...,c, a_{r+s},\underbrace{c...}_{i_s}),
c..., a_{r+s+1},c..., a_{n} \big) 
\end{align*}
The first line is precisely what appear in the equation. In the following line
we expand $m^c$ and $f^c$. Like in the proof of lemma (\ref{prop:Ac_is_ainf}),
we sort the terms according the total number of $c$'s. We get the left hand
side of the $N$th \Ainf-homomorphism relation in $f$ and $m$, applied to $\xi_N \in
A^{\otimes N}$ (where $\xi$ is as in lemma (\ref{prop:Ac_is_ainf})). Applying
the relation we have
\begin{align*}
& \text{LHS} (a_1,...,a_n) 
= \sum_N \sum_{i_1,...,i_r} m^A_r (f_{i_1} \otimes ... \otimes f_{n}) (\xi) \\
&= \sum_N \sum_{i_1,...,i_r} \sum_{k_0,...,k_n} m^A_r (f_{i_1} \otimes ... \otimes
f_{i_r}) (\underbrace{c...}_{k_0}, a_1, c,...,c, a_n,
\underbrace{c...}_{k_n})
\end{align*}
Now, let's also study the right hand side.
\begin{align*}
&\text{RHS} (a_1,...,a_n) \\
&= \sum_{i_1,...,i_r} m^{B,f_*(c)}_r (f^c_{i_1}(a_1,...,a_{i_1}), ...,
f^c_{i_r}(a_{n-i_r},...,a_n)) \\ 
&= \sum_{i_1,...,i_r}\sum_{k_0,...,k_s} m^B_r
(f_{i_1}(\underbrace{c...}_{k_0},a_1,c,...,c,a_{i_1},\underbrace{c...}_{k_{i_1}}), ...,
f_{i_r}(\underbrace{c...}_{k_{s-i_r}},a_{n-i_r},c...,a_n,\underbrace{c...}_{k_s})) 
\end{align*}
where $s = r + \sum i_*$. It is quite clear that these sums agree and thus we
are done.  
\end{proof}


% \begin{lemma}
% $c \in A$ is a MC element if an only if the map $\epsilon: T(A^*) \to k$, given by $\epsiolon(\alpha_1\otimes ... \otimes \alpha_n) = \alpha_1(c)\cdot ... \cdot \alpha_n(c)$, is an augmentation.
% \end{lemma}

\begin{lemma}
Let $f: B \ato C$ and $g: A \to B$, then for any MC element $c \in A$, 
$(f\circ g)^c = f^{g_*(c)} \circ g^c$ and in particular 
$(f \circ g)_* (c) = f_*(g_*(c))$.
\end{lemma}

\begin{proof}
\begin{align*}
&(f_{g_*(c)} \circ g_c)_n (a_1, ..., a_n) \\
&= \sum_{i_1,...,i_r} f^{g_*(c)} ((g_c)_{i_1}(a_1,...,a_{i_1}), ...,
(g_c)_{i_r}(a_{n-i_r},...,a_n)) \\
&= \sum_{i_1,...,i_r}\sum_{j_0,...,j_s} 
f (\underbrace{g_*(c)...}_{j_0 \text{ times}},(g_c)_{i_1}(a_1,...,a_{i_1}), 
g_*(c),...,g_*(c), (g_c)_{i_r}(a_{n-i_r},...,a_n), 
\underbrace{g_*(c)...}_{j_s}) 
\end{align*}
Here the first equality follows from expanding the definition of the
composition. So the sum in $i$ runs over all $i$'s such that $i_1+...+i_r = n$.
The second equality, follows from the expanding of the definition of $f^c$.
Note the notation $\alpha...$, means $\alpha$ is repeated some number of times.

\begin{align*}
&(f \circ g)^c_n (a_1, ..., a_n) \\
&= \sum_{i_1,...,i_n} (f \circ g)_n (\underbrace{c...}_{i_0},a_1,c,...,c,a_n,
\underbrace{c...}_{i_n}) \\
&= \sum_{i_1,...,i_n}\sum_{j_1,...,j_s} 
f_s(
\lefteqn{\underbrace{\phantom{\overbrace{c,...,c}^{\text{apply }g_{j_0}},c...,
c...}}_{i_0}} 
\overbrace{c,...,c}^{\text{apply }g_{j_0}},c..., 
\overbrace{c...,a_1,c,...,c,a_*,c...}^{g_{j_*}}, 
\overbrace{c,...,c}^{g_{j_{*}}}, ... , 
\overbrace{c...,a_n,c...}^{g_{j_*}},..., 
\overbrace{c,...,c}^{g_{j_s}} ) \\
%%
&= \sum_{k_1,...,k_r}\sum_{l_0,...,l_s} 
f (\underbrace{g_*(c)...}_{j_0},(g_c)_{i_1}(a_1,...,a_{i_1}), 
g_*(c),...,g_*(c), (g_c)_{i_r}(a_{n-i_r},...,a_n), 
\underbrace{g_*(c)...}_{j_s}) 
\end{align*}
In the first line we expand $(f\circ g)^c$ according to def. (\ref{def:hom_c}).
And in the second line we expand the composition, where the underbraces indicate
applying $g$ to the indicated elements. On the last line we recognise the
definition of $g_c$ and $g_*$ in the previous line and re-index accordingly.
\end{proof}

\begin{corol}
If $f: A \ato B$ is a finite uncurved \Ainf-isomorphism, 
then for any MC element $c \in A$, 
%
\[ (f^c)^*: H^*(A, m_1^{A,c}) \to H^*(B, m_1^{B,f_*(c)}) \]
is an isomorphism.
\end{corol}

\begin{proof}
We have
\[ A^c \xrightarrow{f^c} B^{f_*(c)} \xrightarrow{g^{f_*(c)}}
A^{g_*(f_*(c))} = A^c,  \]
where the equality follows from the lemma above. Also, by the lemma above 
\[ g^{f_*(c)} \circ f^c = (g \circ f)^c = \id_A^c = \id. \]
Similarly $f^c \circ g^{f_*(c)} = \id_B$. So by lemma
(\ref{prop:Ainf_iso_induced}), the result follows.
\end{proof}

\begin{lemma}
If $f: A \ato B$ is an \Ainf-isomorphism, then $I(A) = I(B)$.
\end{lemma}

\begin{proof}
If $c \in A$ is a MC element, $f_*(c) \in B$ is an MC element and 
\[ (f^c)^* : H^*(A, m_1^{A,c}) \to H^*(B, m_1^{B,f_*(c)}) \]
is an isomorphism, and thus 
\[ P_{A^c} = P_{B^{f_*(c)}}. \]
Therefore $I(B) \subseteq I(A)$. By applying the same argument with
$g^{(f_*(c))}$, \\ $I(A) \subseteq I(B)$ and the result follows. 
\end{proof}

Now, the only thing we have left in order to show that $I(A)$ is a Legendrian
knot invariant. That is % invariance under stabilication.

\begin{lemma}
$I(A) = I(S_i(A)).$
\end{lemma}

\begin{proof}
Let $c \in A$ be a MC element. 
Then $c \oplus 0 \in A \oplus k\<{e_1,e_2}$ is a MC element in $S_i(A)$.
Since clearly $|c\oplus 0| = |c|= 1$ and 
\[\sum_{n\ge 0} m^{S_i(A)}_n((c\oplus 0)...) = \sum_{n\ge 0} m^A_n(c...) = 0.\]

Also, it follows easily from the definition of the stabilization that
$m^{S_i(A), c}_1 (a) = m^{A,c}_n (a)$ for all $a \in A$ and $m^{S_i(A),
c}_1(e_j) = m^{S_i(A)}_1(e_j)$. So, since $m^{S_i(A)}_1(e_1) = e_2$ and
$m^{S_i(A)}_1(e_2) = 0$, 
\[ \im m^{S_i(A), c}_1 = \im m^{A,c}_1 \oplus \im e_2 
\qq{\text{and}} [e_1] = [0] \in H^*(S_i(A), m^{S_i(A),c}_1). \].
So 
\[ H^*\qty(S_i(A), m^{S_i(A),c}_1) = H^*(A, m^{A,c}_1) \]
Thus $I(A) \subseteq I(S_I(A))$. 

Now for the reverse inclusion, suppose $c \oplus (\alpha e_1 + \beta e_2) \in S_i(A)$ is a
MC element for some $c\in A$, $\alpha, \beta \in k$. Then 
\[ 0 = \sum_{n \ge 0} m^{S_i(A)}_n \qty(c \oplus (\alpha e_1 + \beta e_2)...) =
\qty( \sum_{n \ge 0} m^A_n(c...) ) \oplus \alpha e_2. \]
So $\alpha = 0$ and $c \in A$ is a MC element. 

As above it is easy to check that 
\[ H^*\qty(S_i(A), m^{S_i(A),c\oplus \beta e_2}_1) = H^*(A, m^{A,c}_1). \]
Hence $I(A) \supseteq I(S_I(A))$.
\end{proof}

It then follows directly from two lemmas above that.
\begin{them}
$I(A)$ an invariant of stable type and thus, by theorem
(\ref{prop:stable_tame_inv}), a Legendrian knot invariant.
\end{them}

Note that the tame \Ainf-isomorphism condition, in the definition of the stable type,
is not necessary for this theorem to hold. Instead an finite \Ainf-isomorphism
would suffice.

% \subsection{Examples}

% With wich we can conclude that there exist non-isotopic legendrian knots with the same clasical invariances.


