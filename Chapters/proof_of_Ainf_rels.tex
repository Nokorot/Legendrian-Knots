% Proof of A_inf relations

The proof here is essentially the same as that of Theorem 3.3 in chapter 7
[\cite{chekanov02}]. The magare differece is found in the fist subsection, in
which we reduce the algebraic problem to a combinatoiral one.

\subsection{Plan of the proof}

By the linearity of (\ref{eq:a_inf_rel}), it suffuses to show that it is
satisfied when applied on the generators. That is 
\[
\xi_n(a_1,...,a_n) = \sum_{r+s+t = n} m_u(a_1,...,a_r, m_s(a_{r+1, ..., r+s}), a_{r+s+1}, ..., a_n)  = 0,  
\]
for all $a_1, ..., a_n \in \Cc$ (and $u = r+1+t$, like before.)

To show that relation hold, we will show every term in $\xi_n$ occur an even
number of times and thus cancel out, since we are working over $\Z/2\Z$. 

First consider one term in the sum, ie. for a fixed $r,s,t$, such that
$r+s+t=n$. 
Then

\begin{align*}
m_u(a_1,..., a_r, &m_s( a_{r+1}, ..., a_{r+s} ), a_{r+s+1}, ..., a_n) \\
% &= m_u(a_1,...,a_r, \sum_{a\in\Cc} (\#_2 \M^a_{a_{r+1}...a_{r+s}}) a, a_{r+s+1}, ..., a_n )
&= \sum_{a\in\Cc} (\#_2 \M^a_{a_{r+1}...a_{r+s}}) m_u(a_1,...,a_r, a, a_{r+s+1}, ..., a_n ) \\
&= \sum_{a\in\Cc}\sum_{b\in\Cc} (\#_2 \M^a_{a_{r+1}...a_{r+s}}) (\#_2 \M^b_{a_1,...,a_r, a, a_{r+s+1}, ..., a_n}) b \\
&= \sum_{b\in\Cc} Z_{r,s,t}(b,a_1, ..., a_n) b,
\end{align*}
%
where $Z_{r,s,t}(b,a_1,...,a_n)$ is the number of triplets $(u',a,u'')$, such that 
%
\newcommand{\kp}{{k'}}
\newcommand{\kpp}{{k''}}

\begin{itemize}
\item 
$u' \in W^+_\kp(Y,b), \quad a\in \Cc, \quad u'' \in W^+_\kpp(Y,a) $;
\item
$\left( u'(x^\kp_1), ..., u'(x^\kp_{u-1}) \right) = \left(a_1,...,a_r, a, a_{r+s+1}, ..., a_n\right)$
\item
$\left(u''(x^\kpp_1), ..., u''(x^\kpp_s) \right) = \left(a_{r+1},...,a_{r+s}\right)$.
% \item 
% $(a_1,...,a_n) = ( u'(x^\kp_1), ..., u'(x^\kp_r), u''() ) $
\end{itemize}
\figpdf{.4}{splitting}
{An example of a triplet $(u',a,u'') \in U_{1,4,1}(b,a_1,...,a_6)$.}

Here $\kp = u+1$ and $\kpp = s+1$. For an example see figure \pref{fig:splitting}. Let $U_{r,s,t}(b,a_1,...,a_n)$ denote the set of such triplets. Note that for $(r',s',t') \ne (r,s,t)$, 
\[ U_{r,s,t}(b,a_1,...,a_n) \cap U_{r',s',t'}(b,a_1,...,a_n) = \emptyset, \]
so we can conclude that 
\[ \xi_n(a_1,...,a_n) = \sum_{b\in\Cc} Z(b,a_1,...,a_n) b, \] 
where $Z(b,a_1,...,a_n)$ is the cardinality of
\[ U(b,a_1,...,a_n) = \bigsqcup_{r+s+t=n} U_{r,s,t}(b,a_1,...,a_n). \] 
(Note that $r,s,t$ is not in general determined by $(u',a,u'')$, ie. there
might be $(u',a,u'')$ might be in multiple $U_{r,s,t}$. So, by the
\textbf{discrete} sum, we meed that the elements in $U(b,a_1,...,a_n)$ are
indexed by $r,s,t$.)
%
Now in order to prove that $\xi_n$ vanish, we will need to show that
$Z(b,a_1,...,a_n)$ is even. To prove this we will introduce a new space 
$V^+(b,a_1,..., a_n)$ and construct maps 
\begin{align*}
\vphi: U(b, a_1, ..., a_n) \to V^+(b, a_1, ..., a_n), \qquad
\psi_1, \psi_2: V^+(b, a_1, ..., a_n) \to U(b, a_1, ..., a_n),
\end{align*}
such that for all $u \in V^+(b, a_1, ..., a_n)$ and $g \in U(b, a_1, ..., a_n)$;
\[     \phi(\psi_1) = \phi(\psi_2) = u,
\quad  \psi_1(g) \ne \psi_2(g),\text{ and} 
\quad  \psi_1(\phi(g) = g\text{ or }\psi_2(\phi(g) = g. \]
Then clearly $|U(b,a_1,...,a_2)| = 2|V^+(b,a_1,...,a_n)|$ and the result follows.
\footnote{Here $|V|$, means the cardinality of a set $V$}

\subsection{Auxiliary definitions}
\newcommand{\hu}{\hat{u}}
The maps above indicates that $U$ splits into two pars of equal size. So let
\[  U(b,a_1,...,a_n) = U^l(b,a_1,...,a_n) \sqcup U^r(b,a_1,...,a_n). \] 
%where a way of $U^l$, is the set of 
Given a triplet $(u',a,u'') \in U(b,a_1,...,a_n)$, let $S'\subset \Pi_\kp$ be a
neighbourhood of $x^\kp_{r+1}$ and $S'' \subset \Pi_\kpp$ a neighbourhood of
$x^\kpp_0$. Then there are two possibly relative positions of $S_1 = u'(S')$ and
$S_2 = u''(S'')$ near the crossing $u'(x^\kp_{r+1}) = u''(x^\kpp_0) = a$, see figure
\pref{fig:S_rel_pos}
\figpdf{.4}{S_rel_pos}{Relative position of $S_1$ and $S_2$.}

\begin{defn}
We'll define the subsets $U^l(b,a_1,...,a_n)$ and $U^r(b,a_1,...,a_n)$ to be the
sets consisting of triplets $(u',a,u'')$, for which $S_1$ and $S_2$ have the
relative position represented on the left and right side of figure
\ref{fig:S_rel_pos} respectively. 
\end{defn}

We will now define $V^+(b,a_1,...,a_n)$. For each $k \in \N$, let $\Theta_k
\subset \R^2$ be a $k$-sided (curved) polygon, such that the angle of exactly
one vertex is grater then $\pi$. Denote this vertex by $y^k_0$ and the rest 
$y^k_1,...,y^k_{k-1}$ numbered anti-clockwise (see figure
\pref{fig:concave_k_gons}).

\figpdf{.4}{concave_k_gons}{Concave $k$ sided (curved) polygon.}

\begin{defn}
Let $V_k(Y)$ be the set of orientation-preserving immersions $u: \Theta_k \to
\R^2$, such that $\hu(\partial \Theta_k) \subset Y$.

Let $\tilde{V}_k(Y) = V_k(Y) / \Diff_+(\Theta_k)$, the set of non-parametrized
immersions. Here $\Diff_0(\Theta_k)$ is the set of orientation-preserving
diffeomorphisms of $\Theta_k$.
\end{defn}

Consider $\hu \in \tilde{V}_k(Y)$, then for $i>0$, we define the sign of $y^k_i$
in the same way as we did for the immersions of $\Pi_k$ in section
\pref{sect:a_inf_alg_const}. The image of a neighbourhood of $y^k_0$ in
$\Theta_k$ covers three quadrants, we'll say $y^k_0$ is positive (resp.
negative) for $u$ if two of these quadrants are positive (resp. negative).

\begin{defn}
Let $\hu \in \tilde{V}_k(Y)$, then we'll say $\hu$ is admissible if for exactly one $i
\in \{0,...,k-1\}$,
the vertices $y^k_i$ is positive.
% \[ V^+_{k,s}(Y) = \{ u \in \tilde{V}_k(Y) \q| u \text{is admissible and } y^k_s \text{ is positive for } u \}. \]

Let $V^+_{k,s}(Y) \subset \tilde{V}_k(Y)$, be the set of admissible immersions,
such that $y^k_s$ is positive. 
Let $V^+(b,a_1,...,a_{k-1}) \subset V^+_k(Y)$, such that $\hu\in
V^+(b,a_1,...,a_{k-1})$
if and only if $u(y^k_0) = b$ and $u(y^k_i) = a_i$ for all $i \in \{ 1,...,k-1
\}$.
\end{defn}

\renewcommand{\r}{\sigma}
\newcommand{\rp}{\sigma'}
\newcommand{\rpp}{\sigma''}

\subsection{Constructing $\phi$ (Gluing)}
Let $(u', a, u'') \in U_{r,s,t}(b, a_1, ..., a_n)$. The plan is to glue together the 
two polygons $\Pi_\kp$ and $\Pi_\kpp$ (recall that $\kp = u+1 = r+t+2$ and $\kpp
= s+1$) into one concave polygon
$\Theta_{n+1}$, in such a way that the immersions $u'$ and $u''$ combine into
an immersion $\hu \in V^+(Y)$.

Suppose $(u', a, u'') \in U^l(b, a_1, ..., a_n)$. Fix
parametrisations $u'_0: \Pi_\kp \to \R^2$ and $u''_0 : \Pi_\kpp \to \R^2$
representing $u'$ and $u''$ respectively. Acording to \pref{fig:S_rel_pos},
there exists maps $\rp : [0,1] \to \partial
\Pi_\kp$ and $\rpp:[0,1]\to \Pi_\kpp$ such that 
\[ \rp(0) = x^\kp_{r+1} , \: \rpp(0) = x^\kpp_0  \q{\text{and}} u'_0 \circ \rp =
u''_0 \circ \rpp. \]
Choose the maps $\rp$ and $\rpp$ such that the images $\rp([0,1])$, $\rpp([0,1])$
are maximised. Then either $\rp(1)$, $\rpp(1)$ or both is a vertex.

\subsubsection{Case $\kpp>1$:}
(ie. the terms not involving the curvature of the \Ainf-structure $m_0$.)
We'll first consider the case when $\kpp > 1$. Then the above description
looks like one of the three cases showed in figure \pref{fig:glue1}

\figpdf{.8}{glue1}{The thicker line in the middle, indicated the image of $u'
\circ \rp = u'' \circ \rpp$.}

In case (a) $\rp(1) = x^\kp_{r+2}$ and $\rpp(1) \ne x^\kpp_s$, in case (b)
$\rp(1)
\ne
x^\kp_{r+2}$ and $\rpp(1) = x^\kpp_s$ and in case (c) both $\rp(1) = x^\kp_{r+2}$ and
$\rpp(1) = x^\kpp_s$

Define
\[ \Sigma = \Pi_\kp \sqcup \Pi_\kpp / \sim_r, \]
where $\rp(t) \sim_r \rpp(t)$ for all $t\in [0,1]$. Also define 
$\hu : \Sigma \to \R^2$, by
\[  \hu|_{\Pi_\kp} = u' \q\tand \hu|_{\Pi_\kpp} = u''.  \] 

In fact it follows from lemma \pref{prop:height_sum} 
that case (c) is impossible. Indeed, by identifying $\Sigma$ with $\Pi_n$, we
have $\hu \in \tilde{W}_n(Y)$. But since both the positive vertices in the
original immersions are removed by gluing them together with a negative one, all
the vertices are negative, which is impossible, according to the lemma,

In case (a) and (b) showed in figure \pref{fig:glue1}, we have $\Sigma \simeq
\Theta_{n+1}$. Observe that exactly one of the vertices of $\Sigma$ is
positive for $\hu$, namely the one coming from $x^\kp_0$. So $\hu\in V^+(Y)$. In
particular $\hu \in V^+(b,a_1,...,a_n)$. Define $\phi(u',a,'u'') = \hu$. 

\subsubsection{Case $\kpp=1$:}
(ie. the terms coming from the curvature $m_0$.)
Suppose $s=1$. If $
pp(1) \ne x^1_0$, we can proceed like above by gluing
together $\Pi_\kp$ and $\Pi_1$, like in figure (\ref{fig:glue1}a), and define
$\phi(u',a,u'')$ in the same way.

So suppose $
pp(1) = x^1_0$. There are four different cases we need to
consider, see figure \pref{fig:glue2}.

\figpdf{.6}{glue2}{}

If $\rp(1)=x^\kp_{r+2}$, we can glue $\Pi_\kp$ to $\Pi_1$ by identifying
$\rp(t)$ with
$
pp(t)$, like in figure (\ref{fig:glue2}a). But if $\rp(1) \ne
x^\kp_{i+1}$, we need to do some more gluing, we need to glue $\Pi_\kp$ to itself.
Let $\r_\pm: [0,1] \to \partial \Pi_\kp$, such that $\r_+(0) = x^\kp_0$,
$\r_-(0) = \rp(1)$, $u' \circ \r_+(t) = u' \circ \r_-(t)$, for all $t\in[0,1]$ and
such that the images of $\r_+$ and $\r_-$ are maximized. Define 
\[ \Sigma = \Pi_\kp \sqcup \Pi_1 / \sim, \]
where $\rp(t) \sim 
pp(t)$ and $\r_+(t) \sim \r_-(t)$ for all $t\in[0,1]$.

There are three cases to consider, marked by (c), (b) and (d) in figure
\pref{fig:glue2}. Like with case (c) in figure \pref{fig:glue1}, case (d), where
$\r_+(1) = x^\kp_{r+2}$ and $\r_-(1) = x^\kp_{r}$, is impossible, due to lemma
\pref{prop:l_6}, since there are no positive vertices. 

In either of the three cases (a)-(c), $\Sigma \simeq \Theta_{n+1}$. 
So like above, we can, by combining $u'$ and $u''$ along $\rp$, $\rpp$ and
$\r_\pm$, define a new immersion $\hu\in \tilde{V}_k(Y)$. Again, exactly one of the
vertices of $\Theta_{n+1}$ (ie. the original one from $x^\kp_0$) is positive
for $\hu$. Hence $\hu\in V^+(Y)$. In particular $\hu \in V^+(b,a_1,...,a_n)$, so
we may define $\phi(u',a,u'') = \hu$.

Now consider an immersion $w \in U^r(b,a_1,...,a_n)$, then the
construction of $\phi(w)$ will be exactly the mirror image of the
construction above. 

\newcommand{\up}{{u'}}
\newcommand{\upp}{{u''}}
\newcommand{\half}{\frac{1}{2}}

\renewcommand{\k}{{n+1}}

\subsection{Constructing $\psi_1,\psi_2$ (Cutting)}
Let $\hu \in V^+(b,a_1,...,a_n)$. The idea is to cut $\Theta_\k$ into a pair of
polygons diffeomorphic to $\Pi_\kp$ and $\Pi_\kpp$, such that $\kp + \kpp =
n+2$. Then the restriction of $\hu$ to these polygons yield immersions
$\up$ and $\upp$, such that $(\up, a, \upp) \in U(b,a_1,...,a_n)$ for $a =
\upp{x^\kpp_0}$. We will see that there are two ways of doing this cut. One
will define the map $\psi_1$ and the other the map $\psi_2$.

For $i=1,2$, let $\r_i : [0,\half] \to \Theta_\k$, such that
$\r_i(0) = y^\k_0$, $\r_i((0,\half]) \subset \hu^{-1}(Y) \setminus \partial
\Theta_\k$ and such that $\r_1((0,\half]) \cap \r_2((0,\half]) = \emptyset$.
See figure \pref{fig:cut1}. We extend $\r_i$ to smooth immersion $\r_i : [0,1] \to \Theta_\k$, such that, 
$\r_i((0,\half]) \subset \hu^{-1}(Y) \setminus \partial \Theta_\k$ and either
$\r_i(1) \in \partial \Theta_\k$ or $\r_i(1) \in \r_i([0,1))$.
\figpdf{.4}{cut1}{} 
The immersions $\r_i$ are then defined uniquely up to reparametrizations. 

There are four possible configurations of the image of $\r_1$ shown in figure
\pref{fig:cut2}. 
\figpdf{.6}{cut2}{The image of $\r_1$ is represented by the thick line.} 

In each of the four cases $\r_1([0,1])$ divides $\theta_\k$ into two parts,
let's call them $\Sigma_1$ and $\Sigma_2$. In case (a), the positive vertex of
$\Theta_\k$ for $\hu$ is a vertex of either $\Sigma_1$ or $\Sigma_2$, say
$\Sigma_1$.  
In the cases (b)-(d), let $\Sigma_1$ be the polygon on the outside, ie. such
that $\partial \Theta_\k \subset \partial \Sigma_1$, and let $\Sigma_2$ be the
polygon on the inside. We choose $\kp$ and $\kpp$, such that $\Pi_\kp$ (resp.
$\Pi_\kpp$) is diffeomorphic (by an orientation-preserving diffeomorphism) to
$\Sigma_1$ (resp. $\Sigma_2$.) 

Define $\up = \hu|_{\Sigma_1}$ and $\upp = \hu|_{\Sigma_2}$, then 
(quotienting by the set of diffeomorphism) $\up \in \tilde{W}_\kp(Y)$ and $\upp
\in \tilde{W}_\kpp(Y)$. By lemma \pref{prop:l_6.1}, both polygons most have at
least one positive vertex for $\up$ and $\upp$, and since the total number of
positive vertices is 2, we can conclude both $\up$ and $\upp$ are admissible. 
Hence $\up \in W^+_\kp(Y,b)$ and $\upp \in W^+(Y,a)$, where $a =
\upp(x^\kpp_0)$.

In the case (a)-(c), $r,s,t$ are determined uniquely by $(\up,a,\upp)$. But in
case (d), there are two possible values for $r$, say $r_1 < r_2$, such that
\[  (\up, a, \upp) \in U_{r_1,1,t_1}(b,a_1,...,a_n) \qq\tand (\up, a, \upp) \in
U_{r_2,1,t_2}(b,a_1,...,a_n) \] 
($s=1$ in case (d) and $t_i = n - r_i - 1$.)

Define $\psi_1(\hu) = (\up,a,\upp)$, such that $(\up,a,\upp)
U_{r_1,1,t_2}(b,a_1,...,a_n)$ in case (d). 

Following the same construction for $r_2$, case (a)-(c), gives us a new 
$(\up,a,\upp) \in U(b,a_1,...,a_n)$. In case (d), on the other hand, we get the
same triplet as for $r_1$. Using $r_2$, we define $\psi_2(\hu) = (\up, a,
\upp)$, such that in case (d) $(\up,a,\upp) \in U_{r_2,1,t_2}(b,a_1,...,a_n)$
(ie. the larger value of $r$).

This defines $\psi_1$ and $\psi_2$, such that for all $\hu \in
V^+(b,a_1,...,a_n)$, $\psi_1(\hu) \ne \psi_2(\hu)$.

\subsection{Completing the proof}
We need to check that $\hu = \phi(\psi_1(\hu)) = \phi(\psi_2(\hu))$ for all 
$\hu \in V^+(b, a_1,...,a_n)$. This follows easily from the observation that, by
decomposing $\Theta_\k$ into two parts by either $\r_1$ or $\r_2$ follows by
gluing them together, we get back the same polygon.

It remains to show that for each $\tau \in U(b,a_1,...,a_n)$, either $\tau =
\psi_1(\phi(\tau))$ or $\psi_2(\phi(\tau))$. That is we need to check that after
gluing together $\Pi_\kp$ and $\Pi_\kpp$ into a single $\Theta_\k$ followed by
cutting along either $\r_1$ or $\r_2$, we get back the same pair of polygons.
This is quite easy to check. (Note that we also need to check that $r,s$ and $t$
match up, which follows since $r,s,t$ was uniquely determined in case
(a)-(c), in the cutting procedure, and in case (d) there was exactly two cases
corresponding to $\psi_1$ and $\psi_2$.)

